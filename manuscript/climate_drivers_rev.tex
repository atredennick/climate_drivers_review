%---------------------------------------------
% This document is for pdflatex
%---------------------------------------------
\documentclass[11pt]{article}

\usepackage{amsmath,amsfonts,amssymb,graphicx,natbib,setspace,authblk}
\usepackage[running]{lineno}
\usepackage[vmargin=1in,hmargin=1in]{geometry}
\usepackage{rotating}

\usepackage[compact]{titlesec}

\usepackage{graphicx}

\usepackage{times}

\usepackage{setspace}

\usepackage{enumitem}
\setlist{topsep=.125em,itemsep=-0.15em,leftmargin=0.75cm}

\usepackage[textsize=footnotesize,textwidth=2.1cm]{todonotes}
\newcommand{\smalltodo}[2][]
    {\todo[caption={#2}, #1]
    {\begin{spacing}{0.5}#2\end{spacing}}}
% There is nothing \smalltodo{here is a comment} here.
% \todo[inline]{Here is a comment}

\renewcommand{\floatpagefraction}{0.98}
\renewcommand{\topfraction}{0.99}
\renewcommand{\textfraction}{0.05}

\clubpenalty = 10000
\widowpenalty = 10000

\newcommand{\be}{\begin{equation}}
\newcommand{\ee}{\end{equation}}
\newcommand{\ba}{\begin{equation} \begin{aligned}}
\newcommand{\ea}{\end{aligned} \end{equation}}

\def\X{\mathbf{X}}
\def\A{\mathbf{A}}
\def\B{\mathbf{B}}
\def\C{\mathbf{C}}
\def\D{\mathbf{D}}
\def\G{\mathbf{G}}
\def\H{\mathbf{H}}
\def\N{\mathbf{N}}
\def\M{\mathbf{M}}
\def\W{\mathbf{\Omega}}
\def\P{\mathbf{P}}
\def\V{\mathbf{V}}

\title{Detecting and predicting the relationship between climate and population dynamics}

\author[1]{Andrew T. Tredennick\thanks{Corresponding author: atredenn@gmail.com}}
\author[1]{Peter B. Adler}
\author[2]{Giles Hooker}
\author[3]{Stephen P. Ellner}
\author[1]{Brittany J. Teller}
\affil[1]{Department of Wildland Resources and the Ecology Center, Utah State University, Logan Utah}
\affil[2]{Department of Ecology and Evolutionary Biology, Cornell University, Ithaca, New York}
\affil[3]{Department of Biological Statistics and Computational Biology, Cornell University, Ithaca, New York}
\renewcommand\Authands{, and }

\date{Last compile: \today}

\renewcommand{\baselinestretch}{1.2}
\linespread{1.5}

\begin{document}

\maketitle

\vspace{1in} 

\large
{\emph{Running head}: Predicting climate effects}
\normalsize 

\newpage

\renewcommand\linenumberfont{\normalfont\tiny\sffamily\color{gray}}
\linenumbers

\section*{Abstract}
Populations fluctuate in response to internal and external forces.
Quantifying the influence of external forces on population dynamics is a classic focus of population ecology, and it has taken on new relevance as we face the challenge of predicting species' responses to climate change.
However, detecting climate-population growth relationships is difficult and, once detected, the relationship may not improve out of sample prediction --- the gold standard of forecasting.
Here, we review empirical studies to assess how often including climate, or climate-related, covariates improves out of sample prediction of population abundances and dynamics.
We find that...
Drawing on examples from our literature review and our own work (failures?), we outline ways to determine in advance whether one can predict out of sample.
We then use a simulation approach to show how to identify the most efficient ways to improve predictive skill and, in turn, allocate research effort. 
 

\subsection*{Keywords:} Prediction, climate, forecast, population, validation  

\newpage

\section*{Introduction}
Testing more stuff...


\end{document}
